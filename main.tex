\documentclass[10pt,a4paper]{article}
\usepackage[utf8]{inputenc}
\usepackage[english]{babel}
\usepackage{amsmath}
\usepackage{amsfonts}
\usepackage{amssymb}
\usepackage{graphicx}
\usepackage{amsthm}
\newtheorem{definition}{Definition}
\newtheorem{theorem}{Theorem}[section]
\newtheorem{lemma}{Lemma}

\newcommand{\dom}{\mbox{dom}}
\newcommand{\Fun}{\mbox{Fun}}
\newcommand{\Var}{\mbox{Var}}
\newcommand{\Const}{\mbox{Const}}
\newcommand{\Spec}{\mbox{Spec}}
\newcommand{\Types}{\mbox{Types}}
\newcommand{\compat}{\mbox{compat}}
\newcommand{\compatset}{\mbox{compatset}}
\newcommand{\Nat}{\mathbb{N}}
\newcommand{\Pow}{\mathcal{P}}
\newcommand{\Union}{\bigcup}

\author{Georgy Dunaev (georgedunaev@gmail.com)}
\title{%\vspace{-1.5cm}            % Another way to do
Set-Theoretic Foundations of Mathematics}
\begin{document}
\maketitle
\section{Introduction}
This article contain some well-known results which are necessary for reasoning about foundations by means of the same foundations. I hope it will also be implemented in Isabelle 
\section{Recursion theorem}
\begin{definition} Compatible pair of functions. $(compat : i \Rightarrow (i \Rightarrow o))$
\[
\compat(f_1,f_2) : o := \forall x \in \dom(f_1)\cap\dom(f_2). \forall y_1.\forall y_2. (x,y_1)\in f_1 \land  (x,y_2)\in f_2\rightarrow y_1=y_2 
\]
\end{definition}

\begin{definition} A set of functions is compatible when it's pairwise compatible. $(compatset : i \Rightarrow o)$
\[
\compatset(S) : o := \forall f_1\in S.\forall f_2\in S.\compat(f_1,f_2) 
\]
\end{definition}

\begin{lemma}
$\forall x. (\exists y.(x,y)\in f)\longrightarrow \exists u\in f.\exists c\in u.x\in c$
\end{lemma}

\begin{lemma}
$\dom(f)\in \Pow\Union\Union f$.
\[\dom(f)\subseteq \Union\Union f\]
\[\dom(f)\in \Pow\Union\Union f\]
\end{lemma}
\begin{lemma} Let F be a set, then $\{\dom(f) : f \in F \}$ is also set.

\end{lemma}
\begin{theorem}
The union of a compatible set of functions is a function. 
\[
compatset(F) \Longrightarrow \Fun(\bigcup F) \land \dom(\bigcup F) = \bigcup \{\dom(f) : f \in F\}
\]
Proof.
\begin{enumerate}
\item $\Fun(\bigcup F)$
\item $\dom(\bigcup F) = \bigcup \{\dom(f) : f \in F\}$
\end{enumerate}
\[1) Fun()\]
End.
\end{theorem}

\section{Implementation of simply typed $\lambda$-calculus}
%\definition{qwe}
\begin{definition} Special symbols 
\[ \Spec = \{"(",")",","\} \]
\end{definition}

\begin{definition}
%Here is a new definition
Alphabeth \[ \mathcal{A} = \Var\sqcup\Const\sqcup\Spec \]
\end{definition}
\begin{definition}
\end{definition}
\begin{definition}
\end{definition}
\begin{definition}

\end{definition}
\begin{definition}
\[W(X,c):= \]
\end{definition}
\begin{definition}
Function $f$ by recursion:
\[f(0) = \Var\]
\[f(n+1) = f(n)\cup W(f(n))[\Const]\]
\end{definition}
\begin{definition}
Types
\[\Types = \bigcup f[\Nat] \ = \bigcup_{(x\in\Nat)}f(x) \]
\end{definition}
\end{document}